\documentclass[12pt]{exam}

\setlength{\parindent}{0cm} % global indent value

\usepackage{graphicx}
\usepackage{mhchem}
\usepackage{wrapfig}
\usepackage{textcomp}
\usepackage{hyperref}

\pagestyle{headandfoot}
\runningheadrule
\firstpageheader{name:\fillin[][4cm]}{period:\fillin[][1cm]}{Unit 3: Energy}
\runningheader{Unit 3: Energy}
{Class Notes}
{Page \thepage\ of \numpages}
\firstpagefooter{}{}{}
\runningfooter{PACS}{Mr. Maxwell}{page \thepage\ of \numpages}


\begin{document}


\section*{Warm up}

\begin{questions}
    \question How hot does the water get in \textdegree C \fillin[165]
    \question \fillin[explosions] are caused by the thermal shock of \fillin[lava] meeting water.
    \question When lava boils ocean water it creates \fillin[hydrochloric] acid microscopic \fillin[volcanic] shards and super heated \fillin[steam].
    \question What happened to the ice in the lava?
    
    \vspace{2cm}
    
    \question how long have we been studying Mt Etna? \fillin[3500 years]
\end{questions}

\newpage

\section*{Lesson 3.1 Energy}

\begin{questions}
    \question The universe is made of \fillin[matter] and \fillin[energy].
    \question Matter and energy are related by the \fillin[$E=mc^2$] equation.
    \question Energy can be \fillin[potential] or \fillin[kinetic] energy.
    \question \fillin[Temperature] is a measure of the \fillin[kinetic] energy of atoms and molecules.
    \question The \fillin[kinetic] energy of an object is the energy from its \fillin[movement].
    \question The \fillin[Potential] energy of an object is the energy from its \fillin[position].





\end{questions}

\section*{Hot Rocks lab}

\paragraph{lab question} What happens to temperature and energy when you combine hot rocks and cold water?

\subsection*{Materials}

\begin{itemize}
    \item hot rocks (marbles)
    \item glass cup
    \item 50ml beaker
    \item graduated cylinder
\end{itemize}

\subsection*{Procedure}

\begin{enumerate}
    \item Use the graduated cylinder to put 20 mL of water in a 50 mL beaker.
    \item Measure the temperature of the water and the hot rocks using the infrared thermometer. Record both values in the table below.
    \item Pour the water into the cup with the hot rocks.
    \item Wait 1 minute and then measure the temperature of the water and the rocks in the cup. Record the value in the table below.
\end{enumerate}

\subsection*{Data}

\begin{tabular}{|p{4cm}|p{4cm}|p{4cm}|}
    \hline
    \quad & Before combining water and rocks & After combining water and rocks \\
    \hline
    water temperature & \quad & \quad \\
    \hline
    hot rock temperature & \quad & \quad \\
    \hline
\end{tabular}


    
\end{document}