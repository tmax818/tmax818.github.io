\documentclass[12pt]{article}




\title{S}


\begin{document}

\section{Unit 1 - Properties of Matter}

\subsection{Assessment 1.1 - Observing Matter} 

In this assessment students observed three disc magnets
suspended on a rod. The idea was to get them to think about forces on matter that would be
discussed in the unit lessons. This worked moderately well, although the concept of forces with
a direction was difficult for many.

\subsection{Lesson 1.1 - Composition of Matter} 

Atomic and molecular structure of matter and units of mass and length.

\subsubsection{slides}

\begin{enumerate}
    \item Matter is made of \textbf{atoms} that cannot be broken apart.
    Atoms are mostly \textbf{empty} space, but inside atoms there are three kinds of particles:
    Protons and neutrons are in the nucleus of the atom.
    Electrons are outside the nucleus.
\end{enumerate}

\subsection{Lesson 1.2 - Volume, Mass, and Density:} 

Definition of density and relationship to sinking
and floating; practice calculating density.

\subsection{Lesson 1.3 - Forces on Matter}

Properties of gravitational and electromagnetic forces.

\subsection{Assessment 1.2 - Explaining Forces on Matter} 

Students did this assessment after learning about the source and nature of gravitational and electromagnetic forces. It was a good test of their ability to recognize these forces in a simple situation and to explain what they observed.

\subsection{Lesson 1.4 - Temperature and Matter} 

Definition and units of temperature; uses a very
good Phet simulation relating temperature to molecular motion. I referred back to this
simulation many times during the year.

Lesson 1.5 - States of Matter: 

Relationship between the three states of matter and
intermolecular forces; definition of phase changes; definition of a physical change

Lesson 1.6 - Solar Distillation: 

Phase changes in the hydrologic cycle; how a solar still
works. This lesson included the construction of a solar still for each period using
materials that are in the Room 306 back area. The students enjoyed this activity. I took
the still home and reported back on its performance. The previous year each table group
made a still and we put them out on the roof of the building, but this year they would not
let us go on the roof.

Assessment 1.3 - A Water Plan for Your Home and Community: 

This assessment was directly
related to the lesson on solar stills and was an attempt to get the students to apply this
information qualitatively and quantitatively to their home and community within a scenario of
scarce fresh water.


\section{Unit 2 - Combustion}

\subsection*{Assessment 2.1 - Observing Combustion} 

Students observe and reflect on a burning candle.
This assessment continues the theme for the first assessment in which they are asked to
describe what they see and also speculate about what they cannot see. One possible extension
of this in the amazing parte would be to ask them to formulate a question about what they see
that involves what they cannot see.

\subsection{Lesson 2.1 - Computing the Energy in Food} 

Units of energy; computing the energy per mass in food using food labels; displaying the results as a histogram of sticky notes on a number line on the back white board.

\subsection{Lesson 2.2 - Bio-fuels Lab}: 

Lab materials and procedure copied from a PowerPoint to a
single sheet in their notebook; data on water volume and temperature and nut mass. Reflection on kinds and sources of error.

\subsection*{Assessment 2.2 - Calculating the Energy Increase in Water} 

This assessment uses the data from the bio-fuel lab and asks students to calculate the energy increase in their mass of water
using the specific heat of water and the measured temperature increase. Students then calculate the energy going into the water per mass of nut burned and compared it to the energy
per mass of food measured in Lesson 2.1. Students were asked to defend a claim about whether all the energy from the nut went into the water. This was a difficult assessment for
many students, even though an example calculation was provided. They are very challenged by math calculations. They were also challenged by having to use concepts such as less than or
greater than in making their claim.

\subsection{Lesson 2.3 - Combustion Conference}

Individual, group, and class responses to three questions about the combustion lab. I did this using a Fishbowl routine in which
representatives from each table came to a central table to discuss the questions. I used the sentence starter sheets to guide the discussion. I provided the wording for the class
response.

\subsection{Lesson 2.4 - Combustion Video Questions} 

Students watched a video about combustion and filled in words from a word bank into statements taken from the video.

\subsection*{Assessment 2.3 - Real World Combustion Project} 

Each student chooses a fuel and does research on the properties of the fuel. I would say that the Amazing question about the
relevance of the fuel to home, community, culture, or country needs to be more well-defined and a bit more demanding. As it is, students dash off a couple of sentences.

\section{Unit 3 - Energy}

Note this is a large unit with several distinct parts, including heat transfer,
plate tectonics, and energy systems

\subsection*{Assessment 3.1 - Observing Lava Flowing into the Ocean} 

Students watch a video of lava flowing into the ocean and respond with their observations of what they can and cannot see.

\subsection{Lesson 3.1 - Hot Rocks Mini-lab} 

In this lab students observe cold water being poured over heated marbles (>200C) and to measure the temperature of the marbles and water before and after they are combined. We use the IR thermometer that is in one of the lower supply drawers in the front table. The dishes and marbles are in the cabinets in the
storeroom across the hall. I bought a toaster oven from home to heat the marbles.

\subsection{Lesson 3.2 - Hot Rocks Discussion} 

We did this using the Fishbowl routine with sentence starters.

\subsection{Lesson 3.3 - Heat Transfer Lab}

Students heat up water using radiation, conduction, and
convection. As part of the data analysis they calculate how much energy was transferred
to the water.

\subsection{Lesson 3.4 - Heat Transfer Reflection}

After the students try to answer the question about how heat was transferred for each case I use red plastic chips to illustrate how the transfer takes place. Then I show them the class answer for them to copy. This is not a perfect way for them to learn how each kind of heat transfer works, but it seems to get
the point across. A possible tweak would be to make the names of the three modes of heat transfer more prominent given how much we are going to refer to them.

\subsection{Lesson 3.5 - Heat Transfer Conference}

After students try to draw their individual responses each table uses a felt board and felt elements to draw what they think the
answer is. This exercise is difficult because all of a sudden we are talking about more than one mode of transfer happening for each case. I walk around helping the tables make their diagrams and then each table shares its diagram. Finally, I project the class answer for them to copy.

\subsection{Lesson 3.6 - Heat Transfer Video}

This is a somewhat creepy but effective video for helping them remember the essential elements of each mode of transfer.

\subsection{Lesson 3.7 - Plate Tectonics Video}

This video is a good introduction to plate tectonics.

\subsection{Lesson 3.8 - Convection Remembrance, Minilab, Video, and Reflection}

The remembrance is what happened in the heat transfer lab with convection. The Mini-lab is a pyrex baking pan on a hot plate with a light above and potassium permanganate crystals
dropped in to show the convection pattern after about 5 minutes of heating. I used to let the students do the lab, but this year I just demonstrated it at the front table using the
data camera. The movie is part of the plate tectonics video. The reflection should comment on how each part of the lesson shows heated material rising upward carrying
energy.

\subsection{Lesson 3.9 - Dynamic Earth Reading} 

The lesson has a pre-read part involving vocabulary and statements about plate tectonics the student agree or disagree with. Then they do the reading and provide responses to selected sentences. The last thing they are supposed to do is go back to the statements and correct any agreement or disagreement that is wrong and for those statements that are wrong they should write
what is correct. For some reason students find it hard to understand what to do in this last part.

\subsection*{Assessment 3.2 - Plate Tectonics} 

For the Must Have they make and annotate a drawing showing the basic parts of the Earth's interior and how heat is moved. For the Amazing they should say how plate tectonics has affected life on Earth.

\subsection{Lesson 3.10 - Forms of Energy:} 

This is the beginning of the last part of the energy unit.
Six kinds of energy are identified - three are forms of kinetic energy and three are forms of potential energy. The scenarios have been a good way to lock in their understanding.

\subsection{Lesson 3.11 - Energy Systems}

This is the only place that conservation of energy is discussed. The Phet app is excellent for visualizing energy systems.

\subsection{Lesson 3.12 - Energy System Mini-lab}

This is a fun hands-on activity that uses different material. Teapots on a hotplate; large lights as radiation source; batteries; solar panels; propellers; spools; generator/motors that the spools and propellers go on; small lights; LED lights, pulleys, weights, frictionless cars. Students construct enough energy systems, usually three but sometimes two, that have all six kinds of energy.

\subsection*{Assessment 3.3 - Energy Systems Project} 

This project asks students to conceive of a real-
world energy system and to answer questions about their system.
Energy Reflection: This is a fun way to end the semester by completing an artistic work
(drawing, clay sculpture, poem, etc.) that expresses how the student thinks about energy. Clay
has been the most popular media, so if you do this you should buy a lot of clay from Amazon or
Staples. This is what I bought:
You can see the results of the last two years of this reflection at


\section{Unit 4 - Atoms and Elements}

\subsection*{Assessment 4.1 - Mystery Tubes} 

This assessment introduces the idea of trying to figure out
what is inside something when you cannot see what is inside. The mystery tubes and the materials for making model tubes are in the storeroom across the hall.

\subsection{Lesson 4.1 - Atomic Model Research} 

Each student is assigned one of the five atomic models and does research on that model using the graphic organizer. Students at the same table have different assignments. Then students meet in groups by the model they were assigned and make a slide presentation and a poster.

\subsection{Lesson 4.2 - Atomic Model Timeline}

Students take notes on the presentations of each
model, noting the claims made by each model.

\subsection{Lesson 4.3 - Element Property Lab: }

Students make measurements on seven different
element samples.

\subsection{Lesson 4.4} 

- Students practice categorizing different objects and then try to categorize
the element samples.

\subsection{Lesson 4.5 - Periodic Table Notes: }

Introduces the essential features of the periodic
table.

\subsection{Lesson 4.6 - Periodic Table Practice: }

Students practice identifying the properties of
elements using the periodic table notation.

\subsection*{Assessment 4.2 - Build an Atom}

Students us a Phet app to practice building atoms with specific
properties and identifying element isotopes.

\subsection{Lesson 4.7 - Bohr Electron Diagram Notes} 

Introduces the properties of a Bohr atom and how it is represented in an electron diagram.

\subsection{Lesson 4.8 - Drawing Bohr Electron Diagrams} 

Students practice drawing the diagrams
for elements 1 through 18. This is a super important lesson because we refer to it a lot in
future lessons.

\subsection{Lesson 4.9 - Electronegativity} 

Students add electronegativity values to the Lesson 4.8
diagrams and then discuss the trends in electronegativity values.

\subsection*{Assessment 4.3 - Adopt an Atom}

Each student is assigned a different element and does research to identify the properties of that element.

\section{Unit 5 - Bonding and Material Properties}

\subsection*{Assessment 5.1 - Observing a Paper Towel and Water}: 

Students observe paper towel lifting water from one cup to another.

\subsection{Lesson 5.1 - Properties of Water Lab} 

This is a fun, but logistical, lesson where students observe water flowing down a cord, sticking to a penny, mixing with oil and alcohol, and dissolving salt.

\subsection{Lesson 5.2 - Properties of Water Lab Discussion}

After the concepts of cohesion and adhesion are introduced students try to explain what they saw in the lab.

\subsection{Lesson 5.3 - Lewis Dot Diagrams and Ion Formation} 

Student learn about these two concepts.

\subsection{Lesson 5.4 - Practice forming Ions} 

Students use the notation to show how anions and cations are formed.

\subsection{Lesson 5.5 - Bonding Between Atoms} 

Introduces the concept of “happy” atoms with full shells. Uses the excellent Happy Atoms that are in the back 306 room in conjunction with a bonding game.

\subsection*{Assessment 5.2 - Adopt a Molecule} 

Each student is assigned a different molecule and does research to identify the properties of that molecule.

\subsection{Lesson 5.6 - Making New Material Lab} 

Students combine calcium chloride and sodium alginate to form solid alginate material and then show that sodium chloride will not do the same thing. There is alginate solution in the refrigerator.

\subsection{Lesson 5.7 - Making New Materials Discussion}

Students use felt boards to discuss why calcium chloride sticks the alginate together.

\subsection{Lesson 5.8 - Polar and Non-Polar Bonds}

Introduces polarity and relates it to material properties

\subsection{Lesson 5.9 - Intermolecular Force Practice} 

A worksheet to help students understand the importance of intermolecular forces.

\subsection*{Assessment 5.3 - Adopt a Material}

Each student is assigned a different material and does research to identify the properties of that material.

\section{Unit 6 - Chemical Reactions}

\subsection*{Assessment 6.1 - Observing a Reaction}

Students observe baking soda and vinegar reacting.

\subsection{Lesson 6.1 - Reaction Mass Conservation Lab} 

Students measure the mass of solids and liquids before and after a reaction.

\subsection{Lesson 6.2 - Reaction Mass Conservation} 

Students use the Happy Atoms to show that the number of atoms of each element in a reaction is conserved.

\subsection{Lesson 6.3 - Mass Conservation in Reactions} 

Students learn about molecular notation and how to use that to determine the number of molecules of reactants and products. Note: students are not retaining what the notation 2H2O means in terms of the number of each atom and also they are not retaining the idea that H2O represents a molecule. Anything that can be done to solidify this idea for them will help in the
next lessons.

\subsection{Lesson 6.4 - Reaction Mass Conservation Computations} 

Students learn how to compute the mass of reactants and products. Note: I do not use moles at all in my lessons.

The word “mole” appears only once in the Three Course Model writeup for
Chemistry.

\subsection{Lesson 6.5 - Reaction Mass Conservation Practice} 

More practice showing that total mass is conserved.

\subsection*{Assessment 6.2 - Reaction Mass Conservation} 

Student are assigned one of four reactions for which they show that total mass is conserved.

\subsection{Lesson 6.6 - Battery Mini-lab}

Students make batteries out of potatoes or lemons (about 30 of each is enough - they can be reused) and measure the voltage and show that by adding elements in series the voltage goes up enough to lite a small LED.

\subsection{Lesson 6.7 - The Lemon and Potato Battery Explained}

Students watch a video about the invention of the battery and then take notes on how it works. Note: in my notes the electrons end up reacting with hydrogen ions; some sources have the electrons combining with copper ions in solution to reform solid copper.

\subsection{Lesson 6.8 - Reactions and Energy} 

The Happy Atoms are used in conjunction with notes to show that in a reaction energy is first added to break up the reactants and then emitted when the products are formed. The concept of exothermic and endothermic reactions is introduced.

\subsection{Lesson 6.9 - Ocean Acidification Mini-lab}

This lab uses a Phet app to define pH. Students show that vinegar is acidic and will dissolve shells. Students show that adding $CO_2$ to water makes it more acidic.

\subsection{Lesson 6.10 - Ocean Acidification Video}

Provides more information about ocean acidification.

\subsection{Assessment 6.3 - Adopt a Reaction} 

Each student is assigned a different reaction and does research to identify the properties of that reaction

\section{Unit 7 - Climate Change}

subsection*{Assessment 7.1 - Climate Change Reflection} 

Students interpret what four graphs show in terms of climate change.

\subsection{Lesson 7.1 - Climate Change Videos}

Students watch videos on each of the four climate change topics.

\subsection*{Assessment 7.2 - Climate Change Mini-quizzes}

Students use material provided to pass a mini-quiz on each of the four climate change topics.

\subsection{Lesson 7.2 - Climate Change Simulations} 

Students us a simulation app to show how different emission scenarios affect the severity of climate change.

\subsection*{Assessment 7.3 - Climate Change Research}

Students define a climate change research question and do research to answer it.



\end{document}